%================================================================
%                    新一篇论文开始
%----------------------------------------------------------------

%----------------------------------------------------------------------
%
%                  论文-填写基本信息
%
%----------------------------------------------------------------------
\qikanming{期刊名}
\nianfen{2016}
\juanshu{25}
\qishu{4}
\zongqishu{99}
\wenzhangdoi{10.11729/syltlx}
\wenzhangbianhao{1672-9897(2014)01-0000-00  }
\zhongtu{(V211.751)}
\wenxianbiaozhi{A}

\biaoti{标题}
\fubiaoti{}

\authorinput{莫某某; 李~~四; 赵~~五} %作者间用英文分号隔开
\schoolinput{中国洛阳电子装备试验中心~~河南洛阳~~471000;中国电子装备研究中心~~北京~~471000}                        %单位间用英文分号隔开
\includinput{1,2; 1; 1}              %各作者所属单位用英文分号隔开,
                                     %一位作者的多个所属单位用英文逗号隔开

\diyizuozhe{第一作者简介:}
\zuozhejianjie{莫某某(1984-),男,硕士,工程师,主要研究方向为通信与通信对抗,Email:hzzmail@163.com}

\zhaiyao{电子战和信号情报系统的“烟囱式问题”严重影响战场能力整合,破解这一问题需要在系统的架构、设计、实现中产生重大转变。转变面临文化方面的挑战包括:从“需要知道信息”到“需要安全共享数据”的理念转变和国防部门采购文化的改变。}
\guanjianci{关键词1 \kongge 关键词2}%多个关键词用\kongge

%----------------------------------------------------------------------
%
%                  论文-正文内容
\tonglan
%\enpaperinfo
\thispagestyle{firstpageps}
\pagestyle{main}
\setcounter{section}{-1}
\enlargethispage{-1.0\baselineskip}
\begin{refsection}
%\begin{multicols}{2}
%\makeatletter
%\let\mcnewpage=\newpage
%\newcommand{\TrickSupertabularIntoMulticols}{%
%  \renewcommand\newpage{%
%    \if@firstcolumn
%      \hrule width.918\linewidth height0pt
%       \columnbreak
%    \else
%      \mcnewpage
%    \fi
%  }%
%}
%\makeatother 
%----------------------------------------------------------------------

\section{引言}
美军通信与通信对抗战术战法

\section{内容}

\subsection{一些测试}
美军通信与通信对抗战术战法\cite{詹广平2013-8-10}

\section{结论}
特种部队潜入破坏我物理通信网

%----------------------------------------------------------------
{
\tolerance=5000
\printbibliography[heading=subbibliography,title=参考文献]
}
%\end{multicols}
\end{refsection}
\clearpage 
%一篇论文结束
%
%----------------------------------------------------------------
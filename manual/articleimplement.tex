\section{样式说明及其实现}

%\subsection{表格测试}

%以下这一段测试跨页表格,还存在问题。
%\makeatletter
%\newcount\n %如下代码用于supertabular打补丁
%\n=0
%\def\tablebody{}
%\makeatletter
%\loop\ifnum\n<50
%        \advance\n by1
%        \protected@edef\tablebody{\tablebody
%                \textbf{\number\n.}& shortText
%                 \tabularnewline
%        }
%\repeat
%\makeatother
%
%%\begin{insertlongtab}{示意表二}{tab:testb}
%%
%%\end{insertlongtab}
%
%\begin{center}
%  %\TrickSupertabularIntoMulticols %用于supertabular
% \tabletail{\hline}\tablehead{\hline} %用于supertabular
%\topcaption{This table is split across pages}\label{tab:testb}
%\begin{supertabular}{|p{32pt}|p{57pt}|l|l|}
%\tablebody
%\end{supertabular}
%\end{center}

\subsection{格式实现}

\subsubsection{期刊/文档信息处理}
期刊和文档的信息需要输入并处理:
\begin{texlist}
%=====================================
%     自定义信息和工具命令
%=====================================
%------------------辅助宏包----------
\usepackage{xcolor}   %颜色
\usepackage{etoolbox} %latex编程辅助
\usepackage{ifthen}
%-----------标题,作者等信息输入----
\newcommand{\biaoti}[1]{\def\biaotiudf{#1}}
\def\fubiaotiudf{}
\newcommand{\fubiaoti}[1]{\def\fubiaotiudf{#1}}
%定义作者输入等处理命令
\DeclareListParser*{\forcsvlista}{;}
\def\schoolinput#1{\def\schoollistudf{}%将输入信息传入列表结构中
    \forcsvlista{\listadd\schoollistudf}{#1}}
\def\authorinput#1{\def\authorlistudf{}%
    \forcsvlista{\listadd\authorlistudf}{#1}}
\def\includinput#1{\def\includelistudf{}%
    \forcsvlista{\listadd\includelistudf}{#1}}

\newcounter{includeno}
\newcounter{authorno}
\newcounter{authorna}%all
\newcounter{schoolno}
\newcounter{schoolna}%all

\def\zuozhelist{%用于首页的作者信息
\setcounter{includeno}{0}%利用计数器将作者上标信息存入\listI等命令中
\renewcommand*{\do}[1]{\stepcounter{includeno}%
\expandafter\def\csname list\romannumeral\theincludeno\endcsname{##1}}
\dolistloop{\includelistudf}
\setcounter{authorno}{0}%利用计数器完成列表结构中的作者和上标信息结合
\renewcommand*{\do}[1]{\stepcounter{authorno}%
##1\textsuperscript{\csname list\romannumeral\theauthorno\endcsname}\hspace{1em}}%
\dolistloop{\authorlistudf}}

\def\zuozhelistb{%用于页眉的作者信息
\setcounter{authorna}{0}
\renewcommand*{\do}[1]{\stepcounter{authorna}}%计算总共有多少记录
\dolistloop{\authorlistudf}
\setcounter{authorno}{0}
\renewcommand*{\do}[1]{\stepcounter{authorno}
\ifnumless{\theauthorno}{\theauthorna}{##1\hspace*{1em}}{##1\listbreak}}
\dolistloop{\authorlistudf}}

\def\danweilistb{%用于首页的作者单位信息
\setcounter{schoolna}{0}%利用计数器确定列表结构中数据项总数
\setcounter{schoolno}{0}%利用计数器完成列表结构中的不同位置项的不同格式处理
\renewcommand*{\do}[1]{\stepcounter{schoolna}}%计算总共有多少记录
\dolistloop{\schoollistudf}
\renewcommand*{\do}[1]{\stepcounter{schoolno}%
\ifnumless{\theschoolno}{\theschoolna}{\theschoolno.##1\\}{\theschoolno.##1\listbreak}}%
\dolistloop{\schoollistudf}}

\newcommand{\zhaiyao}[1]{\def\zhaiyaoudf{#1}}
\newcommand{\guanjianci}[1]{\def\guanjianciudf{#1}}
\newcommand{\diyizuozhe}[1]{\def\diyizuozheudf{#1}}
\newcommand{\zuozhejianjie}[1]{\def\zuozhejianjieudf{#1}}

\newcommand{\qikanming}[1]{\def\qikanmingudf{#1}}
\newcommand{\nianfen}[1]{\def\nianfenudf{#1}}
\newcommand{\juanshu}[1]{\def\juanshuudf{#1}}
\newcommand{\qishu}[1]{\def\qishuudf{#1}}
\newcommand{\zongqishu}[1]{\def\zongqishuudf{#1}}

\newcommand{\zhongtu}[1]{\def\zhongtuudf{#1}}
\newcommand{\wenxianbiaozhi}[1]{\def\wenxianbiaozhiudf{#1}}
\newcommand{\wenzhangdoi}[1]{\def\wenzhangdoiudf{#1}}
\newcommand{\wenzhangbianhao}[1]{\def\wenzhangbianhaoudf{#1}}
\end{texlist}

\subsubsection{字体和页面布局设置}
通用字体和页面布局设置如下:
\begin{texlist}
%======================================
%     字体和页面布局
%======================================
%-----------------英文字体-----------
\usepackage{fontspec}
%----英文字体第零套:用new roman---------
%不用任何设置
%----英文字体第零1套:用new roman+dejavu-
\setmainfont[Ligatures=TeX]{DejaVu Serif}
\setsansfont[Ligatures=TeX]{DejaVu Sans}
\setmonofont[Ligatures=TeX]{DejaVu Sans Mono}
%----英文字体第零2套:用couriernew_libertine-
%\setmonofont[Ligatures=TeX]{Courier New}
%\setmainfont[Ligatures=TeX]{Linux Libertine O}
%\setsansfont[Ligatures=TeX]{Linux Biolinum O}
%----英文字体第一套:衬线字体用new roman非衬线和等宽用Consolas---
%\setsansfont[Path="fonts/"]{PTSansNarrow.ttf}
%\setmonofont[Path="fonts/"]{PTSansNarrow.ttf}
%\setmonofont[Path="fonts/"]{YaHei.Consolas.ttf}
%----英文字体第二套:衬线字体用new roman非衬线和等宽用SourceCodePro---
%\setsansfont[Path="fonts/"]{SourceCodePro-Regular.otf}
%\setmonofont[Path="fonts/"]{SourceCodePro-Regular.otf}
%----英文字体第三套:全部用linuxLibertine-
%\setmainfont{LinLibertine_R.otf}[
%Path="fonts/",
%BoldFont=LinLibertine_RB.otf,
%ItalicFont=LinLibertine_RI.otf,
%BoldItalicFont=LinLibertine_RBI.otf,
%SlantedFont = LinLibertine_aRL.ttf,
%BoldSlantedFont = LinLibertine_aBL.ttf,
%SmallCapsFont = LinLibertine_aS.ttf
%]
%\setsansfont{LinBiolinum_R.otf}[
%Path="fonts/",
%BoldFont=LinBiolinum_RB.otf,
%ItalicFont=LinBiolinum_RI.otf,
%BoldItalicFont=LinBiolinum_RBO.otf,
%]
%\setsansfont{LinLibertine_M.otf}[
%Path="fonts/",
%BoldFont=LinLibertine_MB.otf,
%ItalicFont=LinLibertine_MO.otf,
%BoldItalicFont=LinLibertine_MBO.otf,
%]
%--------------中文字体------------
\usepackage[zihao=5]{ctex}
\usepackage{xeCJKfntef}
%--------中文字体第零套:采用windows字体-
\setCJKmainfont[BoldFont={SimHei},ItalicFont={KaiTi}]{NSimSun}
\setCJKsansfont{SimHei}
\setCJKmonofont{FangSong}
\setCJKfamilyfont{zhsong}{NSimSun}
\setCJKfamilyfont{zhhei}{SimHei}
\setCJKfamilyfont{zhkai}{KaiTi}
\setCJKfamilyfont{zhfs}{FangSong}
\setCJKfamilyfont{zhli}{LiSu}
\setCJKfamilyfont{zhyou}{YouYuan}
%----中文字体第一套:采用windows字体---
%\setCJKmainfont[Path="fonts/",
%BoldFont={simhei.ttf},
%ItalicFont=simkai.ttf]{simsun.ttc}
%\setCJKsansfont[Path="fonts/"]{simhei.ttf}
%\setCJKmonofont[Path="fonts/"]{simfang.ttf}
%\setCJKfamilyfont{zhsong}[Path="fonts/"]{simsun.ttc}
%\setCJKfamilyfont{zhhei}[Path="fonts/"]{simhei.ttf}
%\setCJKfamilyfont{zhkai}[Path="fonts/"]{simkai.ttf}
%\setCJKfamilyfont{zhfs}[Path="fonts/"]{simfang.ttf}
%\setCJKfamilyfont{zhli}[Path="fonts/"]{simli.ttf}
%\setCJKfamilyfont{zhyou}[Path="fonts/"]{simyou.ttf}
%----中文字体第二套:采用adobe字体和思源字体----
%\setCJKmainfont[Path="fonts/",
%BoldFont={AdobeHeitiStd-Regular.otf},
%ItalicFont=AdobeKaitiStd-Regular.otf]{AdobeSongStd-Light.otf}
%\setCJKsansfont[Path="fonts/"]{AdobeHeitiStd-Regular.otf}
%\setCJKmonofont[Path="fonts/"]{AdobeFangsongStd-Regular.otf}
%\setCJKfamilyfont{zhsong}[Path="fonts/"]{AdobeSongStd-Light.otf}
%\setCJKfamilyfont{zhhei}[Path="fonts/"]{AdobeHeitiStd-Regular.otf}
%\setCJKfamilyfont{zhkai}[Path="fonts/"]{AdobeKaitiStd-Regular.otf}
%\setCJKfamilyfont{zhfs}[Path="fonts/"]{AdobeFangsongStd-Regular.otf}
%\setCJKfamilyfont{zhli}[Path="fonts/"]{NotoSansHans-Black.otf}
%\setCJKfamilyfont{zhyou}[Path="fonts/"]{NotoSansHans-Thin-Windows.otf}
%----------------中文字体命令定义---
\renewcommand{\songti}{\CJKfamily{zhsong}} % 宋体
\renewcommand{\heiti}{\CJKfamily{zhhei}}   % 黑体
\renewcommand{\kaishu}{\CJKfamily{zhkai}}   % 楷书
\newcommand*{\kaiti}{\CJKfamily{zhkai}}   % 楷书
\renewcommand{\fangsong}{\CJKfamily{zhfs}} % 仿宋
\renewcommand{\lishu}{\CJKfamily{zhli}}    % 隶书
\renewcommand{\youyuan}{\CJKfamily{zhyou}} % 幼圆
%-----------------数学和其它字体----
\usepackage{pifont} %特殊字体,参考latexfriend
\usepackage{xltxtra} %用于输出tex的特殊文本格式,以及上下标的字符
\usepackage{mflogo,texnames}
%用于输出tex的特殊文本格式,texnames的说明文档没有找到
\usepackage{amsmath}
%使用宏包{美国数学协会数学},有个功能可以将章节和方程号联系起来。
\usepackage{amssymb} %使用宏包{美国数学协会符号}
%----------------页面布局、双栏-----
\usepackage[a4paper,top=3cm, bottom=2.5cm, left=2.5cm, right=2.5cm]{geometry}
\renewcommand{\baselinestretch}{1.35}
\setlength{\headsep}{\baselineskip}%设置页面起始行与页眉间距
\usepackage{multicol}
\setlength{\columnsep}{5mm}
\setlength{\parskip}{2pt plus 2pt minus 2pt} %段落间距设置为0
\renewcommand{\abstractname}{} %摘要名不要
%\usepackage{balance}
\end{texlist}

论文标题等通栏部分字体设置和样式
\begin{texlist}
%--------------首页通栏部分的样式---
\newenvironment*{lrindentlist}
  {\list{}{%
     \setlength{\leftmargin}{1cm}%
     \setlength{\rightmargin}{1cm}%
     \item\relax}}
  {\endlist}
\newcommand{\kongge}{\hspace*{1em}}
\def\temp{}
%\ifthenelse{\equal{\youwuudf}{\temp}}{true}{false}
\newcommand{\tonglan}{
\begin{center}
{\zihao{5}\heiti{Doi: }\wenzhangdoiudf\hfill
\heiti{文章编号: }\wenzhangbianhaoudf}\\
\vspace*{1.0cm}
{\heiti{\zihao{-2}{\biaotiudf}}}\\
{\vspace*{0.2cm}
%\ifx\fubiaotiudf\temp\vspace*{0.2cm}\else{\heiti\zihao{3}\fubiaotiudf}\\\fi
\ifthenelse{\equal{\fubiaotiudf}{\temp}}{\vspace*{0.2cm}}{\heiti\zihao{3}\fubiaotiudf\\}
%\ifdefempty{\fubiaotiudf}{\vspace*{0.2cm}}{\heiti\zihao{3}\fubiaotiudf\\}
\vspace*{0.2cm}}
{\kaiti\zihao{4}\zuozhelist}\\
{\vspace*{0.2cm}\kaiti\zihao{5}(\danweilistb)}
\end{center}
\begin{lrindentlist}
\zihao{5}\heiti{摘\kongge 要~~}\fangsong{\zhaiyaoudf}\par
\heiti{关键词~~}\fangsong{\guanjianciudf}\par
\heiti{中图分类号~~}\zhongtuudf\hfill \heiti{文献标志码~~}\wenxianbiaozhiudf\hspace*{4.5cm}
\end{lrindentlist}
\vspace{0.5cm}}
\end{texlist}


\subsubsection{插图/标题/页面页脚/题注样式}\label{sec:fig:tab:style}
图表插入:
\begin{texlist}
%----------------------插入图片-----
\usepackage{graphicx}
\DeclareGraphicsExtensions{.eps,.mps,.pdf,.jpg,.png}
\DeclareGraphicsRule{*}{eps}{*}{} %%声明不是eps结尾的图片都当成eps处理

\newenvironment{insertfigure}[2]%
{\begingroup\def\figcaptioninfo{#1}\def\figlabelinfo{#2}\noindent}%
{\figcaption{\figcaptioninfo}\label{\figlabelinfo}\endgroup}

%-------------------表格相关-------
\usepackage{multirow} %for excel2latex
\usepackage{booktabs} %for excel2latex
\usepackage{array}
%\usepackage{longtable} %注意longtable无法在multicol中使用
\usepackage{supertabular} %使用supertabular来实现可以跨页的表格

\newenvironment{inserttable}[2]%
{\begingroup\tabcaption{#1}\label{#2}\par\noindent}%
{\endgroup}
\end{texlist}

各级标题和页面页脚样式:
\begin{texlist}
%--------------标题样式和页眉页脚定义--
\usepackage[pagestyles]{titlesec}
%注意titlesec带pagestyles选项之后与fancyhdr宏包有冲突,所以使用要注意
\titleformat{\section}{\zihao{4}\heiti}{\thesection}{1em}{}
\titlespacing*{\section}{0pt}{0.5\baselineskip}{0.5\baselineskip}[0pt]
\titleformat{\subsection}{\zihao{5}\heiti}{\thesubsection}{1em}{}
\titlespacing*{\subsection}{0pt}{0pt}{0pt}[0pt]
\titleformat{\subsubsection}{\zihao{5}\songti}{\thesubsubsection}{1em}{}
\titlespacing*{\subsubsection}{0pt}{0pt}{0pt}[0pt]
%设置页面页脚线存在bug不能直接设置\setheadrule{~pt}默认的宽度会冲掉它
%所以使用下面的两句重定义命令进行修改
\renewcommand\headrule{\setheadrule{1pt}}
\renewcommand\footrule{\setfootrule{0.4pt}}

\newpagestyle{main}{ %%内容奇数页左,偶数页右
\sethead[{\zihao{-5}~\thepage~}]%偶数页左
[{\zihao{-5}~\qikanmingudf}]%偶数页中
[{\zihao{-5}\nianfenudf 年第\juanshuudf 卷}]%偶数页右
{{\zihao{-5}~第\qishuudf 期}\hfil}%奇数页左
{{\zihao{-5}\zuozhelistb :~\biaotiudf ~\fubiaotiudf}}%奇数页中
{{\zihao{-5}~\thepage~}}%奇数页右
\setfoot{}{}{}%
\headrule%
\footrule}%

\newpagestyle{firstpageps}{ %%内容奇数页左,偶数页右
\sethead[{\zihao{-5}~\thepage~~\qikanmingudf}]%偶数页左
[]%偶数页中
[\raisebox{0.4\baselineskip}{\parbox{5cm}{\raggedleft\zihao{-5}\nianfenudf 年第\juanshuudf 卷第\qishuudf 期\\(总第\zongqishuudf 期)}}]%偶数页右
{{\zihao{-5}~\qikanmingudf}}%奇数页左
{}%奇数页中
{\raisebox{0.4\baselineskip}{\parbox{5cm}{\raggedleft\zihao{-5}\nianfenudf 年第\juanshuudf 卷第\qishuudf 期\\(总第\zongqishuudf 期)~~~~\thepage~}}}%奇数页右
\setfoot{%\ ~
%\vspace{0.5cm}
\rule[2\baselineskip]{0.3\linewidth}{0.4pt}
\raisebox{\baselineskip}{\hspace*{-0.3\linewidth}\parbox{\linewidth}{\renewcommand{\baselinestretch}{1.0}%
\large\normalsize\raggedright{\heiti\zihao{6}{\diyizuozheudf}}%
{\kaiti\zihao{6}\zuozhejianjieudf}}}
}{}{}%
\headrule}%
%\footrule}%
\end{texlist}

图表题注样式:
\begin{texlist}
%--------------------图表题注------
\usepackage{subfigure}
\usepackage{ccaption}
\captiondelim{~} %图序图题中间的间隔符号
\captionnamefont{\zihao{-5}\heiti} %图序样式
\captiontitlefont{\zihao{-5}\heiti} %图题样式
\captionwidth{0.5\linewidth} %标题宽度
\changecaptionwidth
%\captionstyle{<style>} can be used to alter this. Sensible values for
%style are: \centering, \raggedleft or \raggedright
\captionstyle{\centering}
%\precaption{\rule{\linewidth}{0.4pt}\par}
%\postcaption{\rule[0.5\baselineskip]{\linewidth}{0.4pt}}
\setlength{\belowcaptionskip}{0.1cm} %设置caption上下间距
\setlength{\abovecaptionskip}{0.1cm}
\setlength{\abovelegendskip}{0.1cm}  %设置legend上下间距
\setlength{\belowlegendskip}{0.1cm}
%注意用中文的图题时,用\\好像出错,而用\newline就没有问题
%采用\\到图列表中时需要用保护命令\protect如:

%浮动体外使用标题,使用\tabcaptionfix命令产生在浮动体外
%与table中默认的caption一致的标题效果
\newfixedcaption{\tabcaption}{table}
\newfixedcaption{\figcaption}{figure}
%\newfixedcaption{\tabcaptionfix}{table}

%双语标题
%\bitwonumcaption[label]{short-1}{long-1}{NAME}{short-2}{long-2}
%其中,label为标签,替代普通的\label
%short-1,第一种语言的短标题
%long-1,第一种语言的长标题
%name,第二中语言的图序名称,如:英语的fig,中文的图
%short-2,第二种语言的短标题
%\bionenumcaption,与\bitwonumcaption类似,差别在于第二种语言在目录中不列序号
%\bicaption[label]{short-1}{long-1}{NAME}{long-2},
%也是类似,差别在于第二中语言不列入目录中
%\midbicaption,类似于precaption和postcaption,双语标题,其本质是两次调用caption
%使用\midbicaption命令就是在调用第二次caption前重设precaption和postcaption
%比如\precaption{\rule{\linewidth}{0.4pt}\par}\postcaption{}
%\midbicaption{\precaption{}\postcaption{\rule{\linewidth}{0.4pt}}}

%续表的双语标题使用命令:
%\bicontcaption{long-1}{NAME}{long-2}

%注意ccaption与longtable一起使用,longtable需要在ccaption前调用
%且双语的标题略有不同,需要特定的命令如下:
%\longbitwonumcaption
%\longbionenumcaption
%\longbicaption
%\midbicaption

%关于子图的标题,利用宏包subfigure,具体参见说明文档2.5节
%关于浮动体放文章最后的情况,利用宏包endfloat,具体参见说明文档2.6节
\end{texlist}

\subsubsection{参考文献及其字体样式}
参考文献的宏包和样式定义如下:
\begin{texlist}
%-------------------参考文献------
%\usepackage[backend=bibtex8,sorting=none,citestyle=authoryear,bibstyle=alphabetic]{biblatex}
%采用分章的参考文献的快捷方法
\usepackage[backend=biber,%
style=gb7714_2015]{biblatex}%biber
\renewcommand{\bibfont}{\zihao{-5}\songti}
\setlength{\bibitemsep}{2pt}
\defbibheading{bibliography}[\bibname]{%
	\section*{#1}}%
%	\markboth{#1}{#1}}
%注意在实际使用中用titsec宏包的页眉页脚也有点问题,一旦给出markboth就会出现问题
%这里因为bibliography给出了markboth所以出错。因此重定义bibliography以消除问题
%重定义命令中去掉了markboth那一句命令。
%常用样式
%style=trad-plain
%style=trad-unsrt
%style=trad-alpha
%style=trad-abbrv
%style=musuos
%style=nature
%style=nejm  %New England Journal of Medicine
%style=phys  %follows the guidelines of the aip and aps
%bibstyle=publist
%style=science  %follows the guidelines of the journal Science

%style=ieee % follows the guidelines of the ieee.
%style=ieee-alphabetic
%bibstyle=manuscripts,otheroption
%style=chem-acs %American Chemistry Society journals.
%style=chem-angew %Angewandte Chemie Chemistry – A European Journal.
%style=chem-biochem %Biochemistry and a small number of other American Chemistry Society journals.
%style=chem-rsc  %Royal Society of Chemistry journals.
%style=authortitle-dw
%style=footnote-dw
\end{texlist}

参考文献的数据库文件输入在导言区中使用命令:
\begin{texlist}
%参考文献库添加
%论文1的参考文献的bib文件
\addbibresource{paperone.bib}
%论文2的参考文献的bib文件
\addbibresource{papertwo.bib}
\end{texlist}

\subsubsection{超链接}
超链接相关设置:
\begin{texlist}
\usepackage[CJKbookmarks,%
colorlinks=true,%
bookmarksnumbered=true,%
pdfstartview=FitH,%
linkcolor=magenta,%
anchorcolor=magenta,%
citecolor=magenta]{hyperref}   %书签功能,选项去掉链接红色方框,
%linkcolor=black,linkcolor=green,blue,red,cyan, magenta,
%yellow, black, gray,white, darkgray, lightgray, brown,
%lime, olive, orange, red,purple, teal, violet.
%下面这两句保证多篇文章一起排时的超链接正确性
\newcounter{Hsection}
\preto\section{\stepcounter{Hsection}}
\usepackage{titleref} %标题引用
\end{texlist}

\subsubsection{源代码环境}
如下利用listings宏包定义了三种常用源代码环境分别是latex,fortran和mathematica语言的源代码环境。
\begin{texlist}
\usepackage{xcolor}
\usepackage{listings}
\lstnewenvironment{texlist}%
{\lstset{% general command to set parameter(s)
%name=#1,
%label=#2,
%caption=\lstname,
linewidth=\linewidth,
breaklines=true,
%showspaces=true,
extendedchars=false,
columns=fullflexible,%flexible,
frame=none,
fontadjust=true,
language=[LaTeX]TeX,
%backgroundcolor=\color{yellow}, %背景颜色
%numbers=left,
%numberstyle=\tiny\color{red},
basicstyle=\scriptsize, % print whole listing small
keywordstyle=\color{blue}\bfseries,%\underbar,
% underlined bold black keywords
identifierstyle=, % nothing happens
commentstyle=\color{green!50!black}, % white comments
stringstyle=\ttfamily, % typewriter type for strings
showstringspaces=false}% no special string spaces
}
{}

\lstnewenvironment{fortranlist}%
{\lstset{% general command to set parameter(s)
%name=#1,
%label=#2,
%caption=\lstname,
linewidth=\linewidth,
breaklines=true,
%showspaces=true,
extendedchars=false,
columns=fullflexible,%flexible,
frame=none,
fontadjust=true,
language=[08]Fortran,
%backgroundcolor=\color{yellow}, %背景颜色
%numbers=left,
%numberstyle=\tiny\color{red},
basicstyle=\scriptsize, % print whole listing small
keywordstyle=\color{blue}\bfseries,%\underbar,
% underlined bold black keywords
identifierstyle=, % nothing happens
commentstyle=\color{green!50!black}, % white comments
stringstyle=\ttfamily, % typewriter type for strings
showstringspaces=false}
} % no special string spaces}
{}

\lstnewenvironment{mathlist}%
{\lstset{% general command to set parameter(s)
%name=#1,
%label=#2,
%caption=\lstname,
linewidth=\linewidth,
breaklines=true,
%showspaces=true,
extendedchars=false,
columns=fullflexible,%flexible,
frame=none,
fontadjust=true,
language=[5.2]Mathematica,
%backgroundcolor=\color{yellow}, %背景颜色
%numbers=left,
%numberstyle=\tiny\color{red},
basicstyle=\scriptsize, % print whole listing small
keywordstyle=\color{blue}\bfseries,%\underbar,
% underlined bold black keywords
identifierstyle=, % nothing happens
commentstyle=\color{green!50!black}, % white comments
stringstyle=\ttfamily, % typewriter type for strings
showstringspaces=false}
} % no special string spaces}
{}
\end{texlist}

\subsubsection{正文中插入的源文件}
首页介绍通讯作者,利用脚注很难实现,所以利用页脚来实现,但为了首页的下页边距与其它页相同,用enlargethispage命令缩短了页面文字高度。同时双栏用multicol宏包实现,但因其中无法放置浮动体,所以定义非浮动体用于插入图表,定义的图表插入环境见第\ref{sec:fig:tab:style}节。

同时为了实现多篇论文可以在一个文档中一起排,所以每篇论文的都用参考文献biblatex宏包的refsection环境包含。所以正文开始插入文件articlemodelpartb.tex的内容为:
\begin{texlist}
\tonglan
\thispagestyle{firstpageps}
\pagestyle{main}
\setcounter{section}{-1}
\enlargethispage{-1.0\baselineskip}
\begin{refsection}
\begin{multicols}{2}
\end{texlist}

正文结尾则打印参考文献并结束论文,插入文件articlemodelpartc.tex的内容为:
\begin{texlist}
\printbibliography[heading=subbibliography,title=参考文献]
\end{multicols}
\end{refsection}
\clearpage
\end{texlist}



